\documentclass[11pt]{article}
\usepackage{fullpage}
\usepackage{setspace}
\usepackage{natbib}
\usepackage{amsmath}
\usepackage{mathtools}
\usepackage[a4paper, margin=2.6cm]{geometry}

\bibpunct{[}{]}{,}{a}{}{;} 

% For multiletter symbols
\newcommand{\Rey}{\ensuremath{\mathrm{Re_\lambda}}}
\newcommand{\R}{\ensuremath{\mathrm{Re}}}
\newcommand{\Rb}{\ensuremath{\mathrm{Re_b}}}
\newcommand\Real{\mbox{Re}} % cf plain TeX's \Re and Reynolds number
\newcommand\Imag{\mbox{Im}} % cf plain TeX's \Im
\newcommand{\Fh}{\ensuremath{\mathrm{F_h}}}
\newcommand{\DD}{\ensuremath{{\cal D}}}
\newcommand{\DT}{\ensuremath{{\cal D}_{tot}}}

\setlength{\parindent}{0in}
\setlength{\parskip}{1em plus0.4em minus0.3em}


\begin{document}
\noindent
{\Large\bfseries Response to Referee \#1}

\vspace*{1em}
\noindent
We appreciate the reviewer's constructive comments, which have proved helpful
in guiding our thinking.  We address each of the referee's comments below and
note the associated changes to the manuscript.  All of the differences between
the original and new versions of the manuscript are indicated by change bars
in the margin.

\noindent{\em
The results of large-scale DNS of decaying stratified turbulent flows at
three different Reynolds numbers are presented. The turbulence is initially
isotropic and the stratification is instantaneously imposed. The
integral length scale is small compared to the computational domain size
which means that the Reynolds numbers are not very high and
no clear stratified turbulence subrange with a k-5/3 is observed. The
flows goes through a stage with strong stratification effects and relatively
weak viscous effects but finally comes into a stage where viscous effects 
determine the dynamics.}

\noindent{\em
The study is thorough but does not try to address new questions. 
Tellingly, the introduction does not mention an aim of the investigation. 
The paper gives a comprehensive picture of the flow when it goes through 
the different stages and that can be a reason for publication in JFM. 
However, there are no other strong arguments for publication and the 
study does not appear to produce new insights. It is also not clear why 
it is interesting to investigate isotropic turbulence suddenly subjected 
to strong stratification. This is not a case that is encountered in 
practice and it is not possible to reproduce by experiments. A clear 
motivation why this case is relevant and interesting should 
at least be given.}

If the paper is considered for publication the following comments also need
to be addressed.

\noindent{\em
- In general it is absolutely not clear in the figures (e.g. figure 3) which 
line/symbol corresponds to which case (with or without stratification) or 
which velocity component etc. The captions of the figures should give much
more and complete information.}

\noindent{\em
- It is not necessary to have both figure 1 and 2 since they both show
the formation of layers due to stratification. I suggest to remove figure 1.}

\noindent{\em
- Define the length scales $L_{ux}$ etc.}

- In figure 4, the straight solid lines, which is the prediction for
the vertical and which for the horizontal length scale?

\noindent{\em
- There is no comment about $Fr_v$ in figure 6.}

\noindent{\em
- Waves/oscillations are exited when the stable stratification is 
suddenly imposed on isotropic turbulence. These play much less a role
in other simulations and experiments. Their effects on the dynamics and
statistics should therefore be discussed in some detail.
}

\bibliographystyle{unsrtnat}
\bibliography{bib}
\end{document}

%<!-- Local IspellDict: en_GB  -->
% vim: set spell spelllang=en_gb:






