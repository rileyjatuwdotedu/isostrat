\documentclass[11pt]{article}
\usepackage{fullpage}
\usepackage{setspace}
\usepackage{natbib}
\usepackage{amsmath}
\usepackage{mathtools}
\usepackage[a4paper, margin=2.6cm]{geometry}

\bibpunct{[}{]}{,}{a}{}{;} 

% For multiletter symbols
\newcommand{\Rey}{\ensuremath{\mathrm{Re_\lambda}}}
\newcommand{\R}{\ensuremath{\mathrm{Re}}}
\newcommand{\Rb}{\ensuremath{\mathrm{Re_b}}}
\newcommand\Real{\mbox{Re}} % cf plain TeX's \Re and Reynolds number
\newcommand\Imag{\mbox{Im}} % cf plain TeX's \Im
\newcommand{\Fh}{\ensuremath{\mathrm{F_h}}}
\newcommand{\DD}{\ensuremath{{\cal D}}}
\newcommand{\DT}{\ensuremath{{\cal D}_{tot}}}

\setlength{\parindent}{0in}
\setlength{\parskip}{1em plus0.4em minus0.3em}


\begin{document}
\noindent
{\Large\bfseries Response to Referee \#4}

\vspace*{1em}
\noindent

We appreciate the reviewer's constructive comments, which have proved helpful
in guiding our thinking.  We address each of the referee's comments below and
note the associated changes to the manuscript.  All of the differences between
the original and new versions of the manuscript are indicated by change bars
in the margin.

\noindent{\em
1. The numerical simulations seem to have been carried out seriously but nowadays,
it seems really old-fashon close-science to say nothing about the code.  It is now
a standard practice to provide a repository of the code used to produce the
numerical results.  How can we convince ourselves that there is no problem in the
code?  In nowadays's science, reproducibility is taken seriously.  Without the
codes, reproducing these results is nearly impossible or at least it would demand
a huge effort.  If you do not want to open the code, I think you must write it
explicitly.}

The pseduo-spectral method has been in use since the 1970's.  The only
variations involve the treatment of the nonlinear terms, the times steppping,
and the dealiasing, all of which are discussed in \S2.  We do not think it
would be useful to the majority of the JFM audience 
to review at length this well-established numerical
techique.

An important point for reproducing the results is that the unstratified cases
are in power law decay as demonstrated in figures 3 and 4.  The details of how
this was done are analagous to the details of how a wind tunnel is physically
constructed to produce isotropic homogeneous turbulence.  We include a few
details for the benefit of those who might want to try something similar, but
if other researchers want to verify our results then they need to create
isotropic homogeneous turbulence in power law decay and then apply a buoyancy
force.  Again, there is an analogy with laboratory experiments; researchers
will normally try to reproduce published results in their own facility, not visit the
original facility and run experiments there.

\noindent{\em
2. The manuscript lacks physical interpretations on what happens in the flow.  When
the stratification is swichted on, approximately half of the energy should be in
poloidal modes, which feel the stratification directly and can behave as waves.
Are waves produced? These waves should be associated with very large vertical
velocity compared to Billant \& Chomaz scaling.  What happen to them?  Then the
toroidal flow should shear itself transferring energy to smaller vertical
scales. What is the typical time for these evolutions? I think other analyses are necessary to study the dynamics of these flow, at least
poloidal/toroidal spectra and spatio-temporal spectra.}

\noindent{\em
3. The conclusions are quite weak. What do we have to retain from your study? What
is rely new?}

\noindent{\em
4.  The last paragraph of the conclusion should be put in the Numerical methods
section. I was thinking about this remark during all my first read.  It would be
much better to have it much before the conclusion.}

\noindent{\em
5. The switch between the name-symbol Buoyancy Reynolds number-$\mathcal{R}$ and
activity parameter-$Gn$ is not clear at all. You first use Buoyancy Reynolds
number-$\mathcal{R}$ for a quantity and then you say we are going to use activity
parameter-$Gn$ for the buoyancy Reynolds number-$\mathcal{R}$ of Brethouwer et
al. (2007).  The name-symbol Buoyancy Reynolds number-$\mathcal{R}$ has been used
in many other studies after Brethouwer et al. (2007). I understand that you want
to change this but then don't use Buoyancy Reynolds number-$\mathcal{R}$ for
another quantity! It is very confusing!
}

$\mathcal{R}$ and $Gn$ are not the same quantity as is clear from their time
evolutions plotted in figure 11.  As identified by \citet{gargett84}, $Gn$
characterizes the scale separation between the Ozmidov and Kolmogorov length
scales.  In contrast, $\mathcal{R}$ was introduced by \citet{riley03} based on
arguments about shear instabilities and its relevance depends on the
applicability of an inertial scaling assumption, that is, an assumption that
$Gn$ is large.  \citet{hebert06b} show that the two quantities differ by a
scalar multiple for a set of flows of the same general type for which $Gn$ is
large.  This may have encouraged \citet{brethouwer07} to define $\mathcal{R}$
in terms of the turbulence length scale, which makes it numerically equivalent
to $Gn$.  A major result of the current manuscript, however, is that the flow dynamics
depends strongly on the magnitude of $Gn$ and, therefore, it is essential that
we consider both quantities.  

We also note that in the stratified flow literature, both $\mathcal{R}$ and
$Gn$ are refered to as `buoyancy Reynolds number' whereas in the oceanography
literature that term is identified with $Gn$.  In an attempt to distinguish
between the two conceptually different quantities, we and others have recently
switched to the symbol $Gn$ and its original name `activity parameter.'  Our
notation may simply add to the confusion since at least some oceanographers
have asked why have changed well-established notation, but they are not
familiar with the stratified turbulence literature in which the same name is
used for two different concepts.


\noindent{\em
Then, the physical difference between your buoyancy Reynolds number-$\mathcal{R}$
and your activity parameter-$Gn$ is not well discussed. Figures 11 and 12 could
tell us about this difference but what do we learn from them?  I think your
discussion in unclear and you do not give argument why you do not consider
differences in Froude number when discussing the bibliography.}


\section*{Other comments}
We have reviewed some of the history of $\mathcal{R}$ and $Gn$ in response to
comment 5, but we are very hesitant to go into great detail in the paper
because they are simply two different quantities unless it is assumed that the
integral and turbulence length scales are equal; figure 7 in
\citet{maffiolli16} is one of many figures showing that this is not the case.
So while $\mathcal{R}$ and $Gn$ have been used interchangeably in some papers,
making a point that they are different is likely to cause more confusion than
it alleviates.  

--------------

\noindent{\em
- The introduction seems to be quite comprehensive but you cite only old
experimental studies about stratified turbulence. Did not you forget more recent
works?}

\noindent{\em
- The Numerical methodology has to be better presented.  Why don't you write the
equation 6.247 of Poe p(2000).  You need to explain why you do this and what it
implies for the decay.  Same thing for the "deterministic forcing schema similar
to that of Overholt \& Pope (1998) (cf. ...)".  Explain at least a bit what it
is. Same thing for the initialisation method.  In order to have a basic idea of
the numerical methods, the reader needs to read four other articles.}

As noted in response above to comment 1, the important point is that the
initial conditions are isotropic and homogeneous and the flow is in power law
decay with the decay constant given.  Again, the analogy with laboratory
experiments is applicable; in the famous 1971 paper by Comte-Bellot and
Corrsin, data is presented to show that the flow is isotropic and homogeneous
but there is minimal discussion of the details of the wind tunnel
construction.  We think that providing the reference as we have is an
appropriate level of detail for a JFM article.

\noindent{\em - I think many figures/captions are not of the quality standard
  for JFM.  Looking at the figure and the caption, the reader should be able
  to understand without looking for a small information in the text necessary
  to understand the figure.  Can you please also improve the figure to have no
  text superposed with lines or symbols.  In some figures, the y limits can be
  changed to better see the curves.}  

Another reviewer has asked for
additional information, such as decay rates, to be superimposed on the
figures. Evidently there is a range of
opinions regarding the size of the figures and how easily they can be
understood without referencing the text.  We have reviewed all the figures,
made several of them larger,

\noindent{\em
- Figure 6: "Horizontal Froude number"... Not only horizontal.}

Corrected.  

\noindent{\em
- Figure 16 and its presentation: I think you should cite Brethouwer et al (2007)
who present a similar figure.  How did you choose the values Fh = 0.6 and Gn = 1?}

\citet{brethouwer07} is insightful and we reference it in our paper.  However,
figures similar to figure 16
have been used at least as far back as talks associated
with \citet{riley03}, including our APS/DFD talk in November 2002, and
probably many years before that.  In fact, Brethouwer et al.\ introduced the
twist of plotting $1/Fr$ rather than $Fr$ so that presenters of talks often
pause to clarify whether the axes are in their original orientation or that of
Brethouwer et al.  Importantly, we plot on the $Fr$-$Gn$ plane, not on the
$Fr$-$Re$, plane because we conclude that $Gn$ is the more important
parameter.  The horizontal lines on the plot are based on the observed changes
in decays rates discussed throughout the paper.

\noindent{\em
- Figure 11: Why don't you normalized with the initial $\mathcal{R}$?}

When stratification is first imposed, the assumptions supporting the relevance
of $\mathcal{R}$ \citet{riley03} do not hold.  They are also do not hold at
late time when $Gn < 1$.  So we do not see why it is
appropriate to scale by this quantity.

\noindent{\em
- You mention that "Of interest is how the dynamics of a flow, initiated at a
somewhat high Froude number, are modified by the presence of stable
stratification, the behaviour of the flow as the stable stratification becomes
dominant, i.e., when Fr < O(1), which is estimated to occur at about one buoyancy
period after flow initiation". It would be good to mention that it depends on the
decay laws introduced in paragraph 3.1.1.}

We are not sure we understand the comment.  That $Fr \sim O(1)$ after one
buoyancy period is based on scaling arguments reviewed in \citet{riley00}.
The decay laws introduced in \S3.1.1 do not include the buoyancy time scale.

\noindent{\em
- It would be easier for the reader to add the exact formula relating D to the
ratio Gn/R in equation (3.2).}

As stated in the sentence following \eqref{3.2}, the exact formula is
$(2\pi)^2 {\cal D}$.  Since ${\cal D}$ is a well-established quantity in the
turbulence literature and the factor of $(2\pi)^2$ arises from our definition
of $Fr$, we think that most readers will recognise \eqref{3.2} as written.  We
could include both, but this seems cumbersome.


\bibliographystyle{unsrtnat}
\bibliography{bib}
\end{document}

%<!-- Local IspellDict: en_GB  -->
% vim: set spell spelllang=en_gb:






