\documentclass[11pt]{article}
\usepackage{fullpage}
\usepackage{setspace}
\usepackage{natbib}
\usepackage{amsmath}
\usepackage{mathtools}
\usepackage[a4paper, margin=2.6cm]{geometry}

\bibpunct{[}{]}{,}{a}{}{;} 

% For multiletter symbols
\newcommand{\Rey}{\ensuremath{\mathrm{Re_\lambda}}}
\newcommand{\R}{\ensuremath{\mathrm{Re}}}
\newcommand{\Rb}{\ensuremath{\mathrm{Re_b}}}
\newcommand\Real{\mbox{Re}} % cf plain TeX's \Re and Reynolds number
\newcommand\Imag{\mbox{Im}} % cf plain TeX's \Im
\newcommand{\Fh}{\ensuremath{\mathrm{F_h}}}
\newcommand{\DD}{\ensuremath{{\cal D}}}
\newcommand{\DT}{\ensuremath{{\cal D}_{tot}}}

\setlength{\parindent}{0in}
\setlength{\parskip}{1em plus0.4em minus0.3em}


\begin{document}
\noindent
{\Large\bfseries Response to Referee \#3}

We are very pleased that the referee recommends publication. We also greatly
appreciate the referee's constructive comments, which have proved very helpful
in guiding our thinking on this topic.  We address each of the referee's
comments below. All of the differences between the original and new versions
of the manuscript are indicated by change bars in the margin.

\noindent
{\em
1. In Figure 3 the decay of the r.m.s. velocity components with time is given for the 3
stratified cases, together with least-squares fits to the data, from which it is clear that
the decay rate is smaller than for HIT. However, the least-squares fits are plotted
without the exponent n being indicated, making it very difficult to understand if
the present DNS results are in quantitative agreement with Davidson's predictions.
It is hinted at in the text that the value of n from the DNS is lower than n = 0.4
as predicted by Davidson and then a value n = 0.28 is reported in the caption of a
later figure, Figure 7. These exponents are of interest and important to appreciate
the accuracy of the theory and/or the difference with the simulations. Similarly to
the non-stratified cases where n is explicitly given, please give values for n for all 3
cases regarding the decay of $u'$ and $v'$. Along the same lines, I would recommend
stating the exponents of $L_h \sim t^m$ and $L_v \sim t^p$ obtained from least-squares fits to
the curves in figure 4 for Case III.
}

\noindent
{\em
2. The evolution of horizontal spectra is considered and they are seen to
evolve significantly from the initial spectrum valid for HIT. A similar and
probably greater evolution is expected for the vertical spectra, which are
expected to become steeper because of the presence of the layers and obey a
form $E_h(k_v) \sim N^2k_v^{-3}$ . It would be interesting to add a plot with the
evolution of the vertical spectra, for example of $E_{xx}^z$, during the early
and intermediate times.
}

\noindent
{\em
3. In \S3.2 concerning the late time results, it is discussed that the flow
enters a viscous-dominated regime and that the vertical lengthscale is expected
to be $L_v \sim L_hRe^{-1/2}$, as proposed by Godoy-Diana et al.\ (2004). It could be
interesting to plot $L_v$ vs.\ $t$
or $L_v Re^{1/2}/L_h$ vs.\ $t$ and check that this relation is verified in the simulations at late
times.
}

\noindent
{\em
4. In the Conclusions and Discussion section the differences with 3 previous
experimental studies which obtained different decay rates are discussed. It is
suggested that the differences may be due to the initial conditions, which in
DNS are HIT while in the experiments consist in a turbulent flow which has
already been modified by the stratification before it becomes fully
developed. While this may be a significant difference, I do not believe it is
the most important one. In the case of the experiments of Praud et al.\ (2005)
the main difference is most probably the Reynolds number since Re is low in
their experiments. As mentioned in \S3.2, this will probably have led to a
decay associated with the viscous-dominated regime with a value of $n$ close to
that of isotropic turbulence, as seen in Figure 17. For this case, this
appears to be a more plausible explanation for the discrepancy.
}


\section*{Minor revisions}

\noindent{\em
* Typos:
\begin{enumerate}
\item `active parameter' (in the Abstract). {\normalfont Corrected}

\item typo in definition of $\hat{N}$ (p.7). {\normalfont\bf I don't see it}

\item $t = 0$ should be
$t = 1$ (caption of Table 1). {\normalfont Corrected}

\item buoyancy flux expression should have no minus sign (end
of p.15). {\normalfont\bf True, but the minus is probably there because we are
  talking about k.e.  We need to consider the wording of the sentence}

\item definition of $L_o$ (p.19), the factor should be of $(2\pi)^2$ (p.21).
{\normalfont\bf There was a typo in the definition, but I don't see why there
  should be a $2\pi$.}

\item it should be
$Fr_h/Fr_v$ (p.27). {\normalfont Corrected}

\item the ratio should be $Rt_G/(15/4)$ (end of p.28). {\normalfont\bf I don't understand}
\end{enumerate}
}

\noindent{\em
* In the Introduction a hypothesis made by Davidson (2010) to arrive at (1.1) has
been missed. It was assumed that $\epsilon \sim u_h^3/L_h$, in stratified turbulence. This should
be stated in the Introduction.}

\noindent{\em
* The gray scale colour map makes it very difficult to understand the value of $Gn$ in
the images in Figure 2. Could a colour map with colours be used instead for clarity?}

{\bf Color doesn't make sense because the variable changes continuosly.
  Several reviewers are not convinced about this figure.  I don't know how
  much the figure 2 adds beyond figure 1}

\noindent{\em
* In \S 3.1.3 please give the definition of $\hat{u}_h$. A brief comment on the evolution of $Fr_v$
and on $Fr_v \approx const$ for the final period of Figure 6 could be added.}

\noindent{\em
* Please give the definition of $L_h$ in Equation (3.2) and which was used in Figure 12.}

{\bf $\boldsymbol{ L_h}$ appears first in the 2nd paragraph of the intro.}

\bibliographystyle{unsrtnat}
\bibliography{bib}
\end{document}

%<!-- Local IspellDict: en_GB  -->
% vim: set spell spelllang=en_gb:






