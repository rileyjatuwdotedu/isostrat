\documentclass[11pt]{article}
\usepackage{fullpage}
\usepackage{setspace}
\usepackage{natbib}
\usepackage{amsmath}
\usepackage{mathtools}
\usepackage[a4paper, margin=2.6cm]{geometry}

\bibpunct{[}{]}{,}{a}{}{;} 

% For multiletter symbols
\newcommand{\Rey}{\ensuremath{\mathrm{Re_\lambda}}}
\newcommand{\R}{\ensuremath{\mathrm{Re}}}
\newcommand{\Rb}{\ensuremath{\mathrm{Re_b}}}
\newcommand\Real{\mbox{Re}} % cf plain TeX's \Re and Reynolds number
\newcommand\Imag{\mbox{Im}} % cf plain TeX's \Im
\newcommand{\Fh}{\ensuremath{\mathrm{F_h}}}
\newcommand{\DD}{\ensuremath{{\cal D}}}
\newcommand{\DT}{\ensuremath{{\cal D}_{tot}}}

\setlength{\parindent}{0in}
\setlength{\parskip}{1em plus0.4em minus0.3em}


\begin{document}
\noindent
{\Large\bfseries Response to Referee \#2}

\vspace*{1em}
\noindent
We are very pleased that the referee recommends publication. We also greatly
appreciate the referee's constructive comments, which have proved very helpful
in guiding our thinking on this topic.  We address each of the referee's
comments below. All of the differences between the original and new versions
of the manuscript are indicated by change bars in the margin.

\noindent
{\em 
1. I don’t understand why the $2 \pi$ factor is included in the definition of the
Froude number. I know the authors tend to do this, but does anyone else? If
not, why bother - it just makes comparisons with other studies more
difficult. On page 2, it is suggested that the $2 \pi$ is included because it is
part of the buoyancy period, and the Froude number is a ratio of time
scales. But one could equally argue that there should be a $2 \pi$ in the eddy time
scale L/U (e.g. in a circular eddy with azimuthal velocity U at radius L). It
just seems arbitrary. Furthermore, it seems to be omitted in some cases ($Fr_h$
in the middle of page 2) and it makes several resulting equations cumbersome,
e.g. for $E_p$ and buoyancy flux on page 15, potential energy dissipation rate on
page 17 (why is it not $(2\pi)^2$ ?), relationship between Gn and R, etc. 
}

We appreciate that we have been unable to persuade the community to adopt the
factor of $2\pi$ in the Froude number.  Nevertheless, we find our definition
to be informative because $Fr \sim O(1)$ is an important theoretical threshold
and an order of magnitude change in $Fr$ significantly changes the flow
because it is $Fr^2$ that appears in the dimensionless momentum equation.  For
example, scaling analysis suggest that $Fr\approx 1$ a buoyancy period after
stratification is imposed, and we use this prediction to verify the current
simulations.    Observing that $Fr \approx 0.16$ after 6.3 buoyancy times, which is the
equivalent statement when the factors of $2\pi$ are omitted, gives the
impression of an emprical result, not the exhibiting of a result predicted by
theory.  

In our opinion, the main difficulty in comparing $Fr$ in the literature is not
the factor of $2\pi$ but the different length scales that are used.  The
literature shows that the horizontal longitidunal integral length $L_h$ is the
dynamically relevant length scale rather than the turbulence length scale $L_t
= u^3/\epsilon$, and that $L_h/L_t$ is a function of $Fr$.  So the common
practice of defining a Froude number defined in terms of $L_t$ makes it a
function of itself and also makes the threshold for strong stratifcation
effects $Fr \ll O(1)$.

So we think that our inclusion of $2\pi$ in the the definition of $Fr$ has
important advantages and presents a very minor problem in comparing Froude
numbers between papers relative to the other challenges in doing so.

{\bf Jim
On page 2 and the other pages of the introduction, many definitions of the
Froude number are used as review the published literature.  On page 2 we have
change modified the notation to show, e.g., that $Fr_h$ scales with $U_h/N
L_h$.}

The definition of $\chi$ on page 17 was in error and has been corrected.

$Gn$ is defined Gibson and is extensively used in the oceanography community
without it being interpreted as $(L_o/L_k)^{4/3}$.  If it is interpreted as
this ratio, then it would be advantageous to include the factor of $2\pi$
because $L_o$ is much smaller than the length that it is hypothesised to
represent, that is, the largest length scale unaffected by buoyancy.
\citet{waite11} and \citet{almalkie12a} recongnise this problem and include
$2\pi$ in the defintion of $L_o$ so that $L_o$ is a physically plausible
length scale in simulations.  However, since the use of $Gn$ predates its
interpretation by \citet{gargett84} as a ratio of length scales, and is
routinely used outside this context, we do not see any inconsistency with our
omitting the factor of $2\pi$ in its definition.

In our opinion, the factor of $2\pi$ is not a significant cause of the confusion
related to $Gn / R \ne 1$ because different length scales and velocity scales
are used in the defintions of $Gn$ and $R$.  Reconciling the two quantities
requires accounting for $L_h/L_t$ and $u/u_h$, both of which vary with, at a
minimum, with Froude and Reynolds number.  In fact we emphasise that $Gn$ and
$R$ are different quantities, as shown by figure 11, not the same quantity
written in terms of different multiplicative constants.


{\bf Jim -- you worked through the derivation and may still have your notes.}

\noindent
{\em
2. Some of the figure captions are difficult to understand. For example, in
figure 3, there are reference lines for all the curves (this should be stated
in caption) but only one is labelled in panels a and b. Also in figure 3, the
caption can be misinterpreted as saying that symbols are for non-stratified
cases in panel d only. It took some work for me to fully understand this
figure. Other captions are lacking in any detail (e.g. figure 17). 
}

We have reworked almost all of the figures and captions in an attempt to make
them as clear as possible.

\noindent{\em
3. There doesn’t seem to be any real reason for including Case IV in table 1,
since it is not discussed or presented. 
}

{\bf Jim -- another review wonders why Case IV is introduced in the
  conclusions.  I think we can just state the result (that it does not matter
  what Fr we start with) and omit Case IV}

\noindent{\em
4. On page 12: how can stratification reduce the dissipation rate in forced
simulations, where the dissipation rate must balance the forcing? I’m not sure
this is a meaningful generic statement, since it will depend on the type of
forcing used. }

In general, forcing can maintain constant energy or inject constant power.  In
the latter case, stratification cannot change the dissipation rate and instead
changes the energy in the flow \citep[c.f.][]{lindborg06a}.  Our forcing
maintains constant energy in a range of wave numbers and so stratification
can, and does, change the dissipation rate.  The text has been changed to
indicate that the simulations referenced used constant-energy forcing.

\noindent{\em
5. It would be interesting to include some energy spectra from the late time
results. Does the horizontal spectrum look like kh−5 at these times, as seen
in forced studies with viscous layering (Waite \& Bartello 2004, Brethouwer et
al 2007)?}

\noindent{\em
6. Page 4, third paragraph: “consistent others” should be “consistent with others”
}

Corrected.

\bibliographystyle{unsrtnat}
\bibliography{bib}
\end{document}

%<!-- Local IspellDict: en_GB  -->
% vim: set spell spelllang=en_gb:






